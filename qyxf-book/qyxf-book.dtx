% \iffalse meta-comment
%
% Copyright (C) 2019 by xjtu-blacksmith
%% -----------------------------------
%
% This file may be distributed and/or modified under the
% conditions of the LaTeX Project Public License, either version 1.3
% of this license or (at your option) any later version.
% The latest version of this license is in:
%
%   http://www.latex-project.org/lppl.txt
%
% and version 1.3 or later is part of all distributions of LaTeX
% version 2005/12/01 or later.
%
% \fi
%
% \iffalse
%<*driver>
\ProvidesFile{\jobname.dtx}
%</driver>
%<*driver>
\documentclass{ltxdoc}
\EnableCrossrefs
\CodelineIndex
\RecordChanges
\usepackage[heading]{ctex}
\usepackage[margin=1in]{geometry}
\begin{document}
\DocInput{\jobname.dtx}
\end{document}
%</driver>
% \fi
%
% \CheckSum{286}
%
% \CharacterTable
%  {Upper-case    \A\B\C\D\E\F\G\H\I\J\K\L\M\N\O\P\Q\R\S\T\U\V\W\X\Y\Z
%   Lower-case    \a\b\c\d\e\f\g\h\i\j\k\l\m\n\o\p\q\r\s\t\u\v\w\x\y\z
%   Digits        \0\1\2\3\4\5\6\7\8\9
%   Exclamation   \!     Double quote  \"     Hash (number) \#
%   Dollar        \$     Percent       \%     Ampersand     \&
%   Acute accent  \'     Left paren    \(     Right paren   \)
%   Asterisk      \*     Plus          \+     Comma         \,
%   Minus         \-     Point         \.     Solidus       \/
%   Colon         \:     Semicolon     \;     Less than     \<
%   Equals        \=     Greater than  \>     Question mark \?
%   Commercial at \@     Left bracket  \[     Backslash     \\
%   Right bracket \]     Circumflex    \^     Underscore    \_
%   Grave accent  \`     Left brace    \{     Vertical bar  \|
%   Right brace   \}     Tilde         \~}
%
%
% \changes{v1.0}{2019/05/03}{初始版本}
%
% \GetFileInfo{\jobname.dtx}
%
% \DoNotIndex{}
% \renewcommand{\filedate}{2019/04/30}
% \renewcommand{\fileversion}{v1.0}
% \newcommand{\QYXF}{\fbox{QYXF}}
% \title{\textsf{\jobname} 文档类: The \textsf{\jobname} class\thanks{This document
%   corresponds to \textsf{\jobname}~\fileversion,
%   dated \filedate.}}
% \author{xjtu-blacksmith \\ \texttt{yjr134@163.com}}
% \date{\filedate}
%
% \maketitle
%
% \begin{abstract}
%   \textsf{\jobname}是由钱院学辅设计的一套\LaTeX{}书籍模板。
% \end{abstract}
%
% \section{简介 Introduction}
%
% \textsf{\jobname}是由钱院学辅设计的一套\LaTeX{}书籍模板。它简洁美观,适用于一般性质
% 书籍内容的排版。
%
% \section{使用 Usage}
% 在\verb|\documenclass|命令中引用\textsf{\jobname}文档类即可。
%
% \subsection{建议}
%\begin{enumerate}
%    \item 不建议在本文档类下使用\textsf{part}级别的标题,
%    因本文档类从未定义过\textsf{part}标题对应的样式和目录样式。
%    此外,针对于\QYXF{}目前所排印的书籍而言,这一级的标题无疑
%    是多余了,无论是对作者还是读者而言。
%\end{enumerate}
%
% \StopEventually{\PrintIndex}
%
% \section{实现 Implementation}
%   以下展示本模板的实现细节。
%
% \subsection{准备工作}
%    \begin{macrocode}
%<*class>
\NeedsTeXFormat{LaTeX2e}[2005/12/01]
\ProvidesClass{qyxf-book}
  [2019/04/28 v1.0 nothing but a skeleton]
\typeout{The book template provided by Qian Yuan Xue Fu}
\LoadClass[10pt,twoside,openany]{book}
%% -------- pkg: 创建关键字属性 --------
\RequirePackage{kvoptions}
\SetupKeyvalOptions{family=QYXF,prefix=QYXF@,setkeys=\kvsetkeys}
\DeclareBoolOption[false]{parskip}
\DeclareBoolOption[false]{smallersize}
\DeclareVoidOption{b5paper}{\kvsetkeys{QYXF}{smallersize}}
\DeclareBoolOption{color}
\DeclareBoolOption[true]{decoration}
\DeclareBoolOption[false]{opensource}
\ProcessKeyvalOptions*
%\DeclareOption*{\PassOptionsToClass{\CurrentOption}{book}}
%    \end{macrocode}
% 这里都是一些基本的定义。
%    \begin{macrocode}
%</class>
%    \end{macrocode}
%
%    \begin{macrocode}
%% -------- pkg: 文本处理 --------
\RequirePackage[fontset=adobe,heading]{ctex}
\setmainfont{Cambria}       % 目前默认使用的serif字体
\setsansfont{Calibri}       % 目前默认使用的sans-serif字体
\setmonofont{Courier New}   % 目前默认使用的等宽字体
\RequirePackage{amsmath}
\RequirePackage{unicode-math}
\setmathfont{Cambria Math}  % 与正文字体相适应的数学字体
\renewcommand{\thefootnote}{\textbf{(\arabic{footnote})}}
%% -------- pkg: 图表处理 --------
\RequirePackage{graphicx}
\graphicspath{{./figure/}{./pic/}{./image}}
\RequirePackage{longtable,booktabs}
\RequirePackage{tikz}
\usetikzlibrary{calc,backgrounds}
\RequirePackage{tcolorbox,tabu}
\tcbuselibrary{listings,theorems,skins}
%% -------- pkg: 数学处理 --------
\newcommand{\exercise}[1]{\tcbox[on line,top=0mm,bottom=0mm,right=0mm,left=0mm]{\bfseries 练习#1}\ }
\newcommand{\note}{\textbf{注记}\ }
\newcommand{\solve}{\textbf{解}\ }
%% -------- pkg: 页面美化 --------
\RequirePackage{geometry}
\ifQYXF@smallersize
  \setlength{\paperwidth}{176mm}
  \setlength{\paperheight}{250mm}
\else
  \geometry{a4paper}
\fi
\geometry{margin=1in}
\RequirePackage{hyperref}
\hypersetup{
    breaklinks,
    unicode,
    linktoc=all,
    bookmarksnumbered=true,
    bookmarksopen=true,
    pdfkeywords={qyxf-book},
    pdfborder={0 0 0},
    linktocpage,
    pageanchor=true
}
\RequirePackage{pgfornament-han}
%% -------- pkg: 标题目录 --------
\ctexset{section={name={\S,}}}
\RequirePackage{titlesec,titletoc}
\titlecontents{chapter}[24pt]{\hspace{-2pc}\filright}
              {\normalsize\bfseries\contentspush{\thecontentslabel\ }}
              {}{\bfseries\titlerule*[8pt]{.}\contentspage}
\titlecontents{section}[20pt]{\filright}
              {\contentspush{\thecontentslabel\ }}
              {}{\titlerule*[8pt]{.}\contentspage}
\titlecontents{subsection}[18pt]{\hspace{2pc}\filright}
              {\contentspush{\thecontentslabel\ }\itshape}
              {}{\titlerule*[8pt]{}\contentspage}
\setcounter{tocdepth}{2}
%% -------- pkg: 页眉页脚 --------
\RequirePackage{fancyhdr}
\fancyhead[EC]{\fangsong\nouppercase\leftmark}
\fancyhead[OC]{\fangsong\nouppercase\rightmark}
\fancyhead[EL,OR]{\bf\thepage}
\fancyhead[ER,OL]{}
\fancyfoot[C]{\textsc{Qian Yuan Xue Fu}\\\resizebox{0.2\linewidth}{1.5ex}{\pgfornamenthan[scale=0.05]{58}}}
\fancypagestyle{plain}{%
\fancyhf{}
\fancyfoot[C]{\textsc{Qian Yuan Xue Fu}\\\resizebox{0.2\linewidth}{1.5ex}{\pgfornamenthan[scale=0.05]{58}}}
\renewcommand{\headrulewidth}{0pt}
\renewcommand{\footrulewidth}{0pt}
}
\setlength{\headheight}{13pt}
\RequirePackage{lastpage}
%% -------- pkg: 列表美化 --------
\RequirePackage{enumerate}
\RequirePackage{pifont}
%% -------- pkg: 附录设计 --------
\RequirePackage{appendix}
\ctexset{appendix={number={\Roman{chapter}}}}
%% -------- pkg: 其他宏包 --------
\RequirePackage[scale=1.1]{ccicons}
%    \end{macrocode}
%
% \subsection{关键字参数生效处理}
% 以下来处理关键字参数对排版效果的影响。
%    \begin{macrocode}
%% -------- kvp: 段距设置 --------
\ifQYXF@parskip
     \setlength\parskip{0.3\baselineskip}
\fi
%    \end{macrocode}
% \subsection{首页制作}
%% -------- sec: 首页设计 --------
%    \begin{macrocode}
\newtoks\subtitle
\newtoks\typo
\newtoks\version
\newtoks\sourcepage
\renewcommand*{\maketitle}{%
\hypersetup{pageanchor=false}
\begin{titlepage}
\phantom{s}
\vspace{3cm}
\begin{center}\large
\begin{tcolorbox}[blanker,borderline west={1mm}{0pt}{black!80},%
left=20pt]
{\fontsize{40pt}{\baselineskip}\selectfont\bfseries\@title}\\[\baselineskip]
{\Huge\itshape\the\subtitle}
\end{tcolorbox}
\begin{tcolorbox}[blanker,borderline west={1mm}{0pt}{black!50},%
left=20pt]
{\Large 作者:\@author\\[0.3\baselineskip]
\@date}
\end{tcolorbox}
\vfill
\texttt{钱学森书院学业辅导中心}\\[0.5\baselineskip]
\textsc{Qian Yuan Xue Fu}\\[0.5\baselineskip]
\includegraphics[width=0.5\textwidth]{name.png}
\end{center}
\begin{tikzpicture}[remember picture,overlay]
  \begin{pgfonlayer}{background}
  \node at ($(current page.east) +(0in,0in)$) {%
  \includegraphics[width=0.8\textwidth]{cover.png}};
  \end{pgfonlayer}
\end{tikzpicture}
\end{titlepage}
\thispagestyle{empty}
\newpage
\hypersetup{pageanchor=true}
\phantom{s}\vfill
\begin{tcolorbox}[title={\bfseries 作品信息}]
\ding{228} \textbf{标题:}\@title{} - \textit{\the\subtitle}\\
\ding{228} \textbf{作者:}\@author\\
\ding{228} \textbf{校对排版:}\the\typo\\
\ding{228} \textbf{出品时间:}\@date\\
\ding{228} \textbf{总页数:}\pageref{LastPage}
\end{tcolorbox}
\begin{tcolorbox}[title={\bfseries 许可证说明}]
\centerline{\tcbox{\ccbyncnd\ \fangsong 知识共享 (Creative Commons) BY-NC-ND 4.0 协议}\\[0.3\baselineskip]}
本作品采用 \href{https://creativecommons.org/licenses/by-nc-nd/4.0/}{\textbf{CC协议}}进行许可。使用者可以在给出\emph{作者署名及资料来源}的前提下对本作品进行\emph{转载},但不得对本作品进行\emph{修改},亦不得基于本作品进行\emph{二次创作},不得将本作品运用于\emph{商业用途}。
\end{tcolorbox}
\newcommand{\opensourceinfo}{根据作者的要求,本作品未公开源码。}
\ifQYXF@opensource
  \begin{tcolorbox}
  本作品已发布于GitHub之上,发布地址为:\\
  \centerline{\the\sourcepage}
  本作品的版本号为\textsf{\the\version}。
  \end{tcolorbox}
\fi
\newpage
}
%    \end{macrocode}
% \subsection{目录制作}
%% -------- sec: 目录设计 --------
%    \begin{macrocode}
\renewcommand\tableofcontents{%
\pagestyle{empty}
\centerline{%
\normalfont\Large\bfseries\contentsname%
\@mkboth{\MakeUppercase\contentsname}{\MakeUppercase\contentsname}
}
\vskip 3ex%
{\setlength\parskip{0pt}\@starttoc{toc}}
\cleardoublepage
\pagestyle{fancy}
\setcounter{page}{1}
}
%    \end{macrocode}
% 到此结束。
% \Finale
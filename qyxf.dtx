% \iffalse meta-comment
%
% Copyright (C) 2019 by xjtu-blacksmith
%% -----------------------------------
%
% This file may be distributed and/or modified under the
% conditions of the LaTeX Project Public License, either version 1.3
% of this license or (at your option) any later version.
% The latest version of this license is in:
%
%   http://www.latex-project.org/lppl.txt
%
% and version 1.3 or later is part of all distributions of LaTeX
% version 2005/12/01 or later.
%
% \fi
%
% \iffalse
%<*driver>
\ProvidesFile{\jobname.dtx}
%</driver>
%<*driver>
\documentclass{ltxdoc}
\EnableCrossrefs
\CodelineIndex
\RecordChanges
\usepackage[heading]{ctex}
\usepackage[margin=1in]{geometry}
\begin{document}
\DocInput{\jobname.dtx}
\end{document}
%</driver>
% \fi
%
% \CheckSum{293}
%
% \CharacterTable
%  {Upper-case    \A\B\C\D\E\F\G\H\I\J\K\L\M\N\O\P\Q\R\S\T\U\V\W\X\Y\Z
%   Lower-case    \a\b\c\d\e\f\g\h\i\j\k\l\m\n\o\p\q\r\s\t\u\v\w\x\y\z
%   Digits        \0\1\2\3\4\5\6\7\8\9
%   Exclamation   \!     Double quote  \"     Hash (number) \#
%   Dollar        \$     Percent       \%     Ampersand     \&
%   Acute accent  \'     Left paren    \(     Right paren   \)
%   Asterisk      \*     Plus          \+     Comma         \,
%   Minus         \-     Point         \.     Solidus       \/
%   Colon         \:     Semicolon     \;     Less than     \<
%   Equals        \=     Greater than  \>     Question mark \?
%   Commercial at \@     Left bracket  \[     Backslash     \\
%   Right bracket \]     Circumflex    \^     Underscore    \_
%   Grave accent  \`     Left brace    \{     Vertical bar  \|
%   Right brace   \}     Tilde         \~}
%
%
% \changes{v1.0}{2019/05/03}{初始版本,提供了基本的功能。}
% \changes{v2.0}{2019/05/10}{将现有的qyxf-book文类与之前的qyxf-note和qyxf-notice
% 文类合并于同一安装文件下,更名为qyxf宏集。}
%
% \GetFileInfo{\jobname.dtx}
%
% \DoNotIndex{}
% \renewcommand{\filedate}{2019/05/10}
% \renewcommand{\fileversion}{v2.0}
% \newcommand{\QYXF}{\fbox{QYXF}}
% \title{\textsf{\jobname} 宏集: The \textsf{\jobname} sets\thanks{本说明文档
%   对应于 \textsf{\jobname}~\fileversion 版本,发布时间为\filedate.}}
% \author{xjtu-blacksmith \\ \texttt{yjr134@163.com}}
% \date{\filedate}
%
% \maketitle
%
% \begin{abstract}
%   \textsf{\jobname}是由钱院学辅(QYXF)设计的一套\LaTeX{}书籍模板,由钱院学辅学研部
%    维护。该模板主要用于正规、中等以上容量资料的排版,提供有多个可选项。目前该模板不计划
%    对一般用户开放(和优化)。
% \end{abstract}
%
% \section{简介}
%
% \textsf{\jobname}是由钱院学辅设计的一套\LaTeX{}书籍模板。它简洁美观,适用于一般性质
% 书籍内容的排版。
%
% 目前,\textsf{\jobname}主要用于正规、中等以上容量资料的排版,提供有多个可选项。目前该模板
% 不计划对一般用户开放(和优化)。
%
% \section{使用}
% 在\verb|\documenclass|命令中引用对应的文档类即可。
%
% \subsection{qyxf-book 文类选项}
% 本文类提供了以下几个可选项:
%
% \begin{description}
% \item[\textsf{b5paper}] 使用比\textsf{a4paper}更小的B5纸张尺寸(250mm*176mm)
% 进行页面设计。默认使用\textsf{a4paper}。
% \item[\textsf{parskip}] 增加使段落间更为宽敞的的段距。默认关。
% \item[\textsf{color}] 使用有颜色的主题进行排版。默认关。(目前此功能尚未实现)
% \item[\textsf{decoration}] 增加一些页面装饰。默认开。(目前尚未实现关闭功能)
% \item[\textsf{opensource}] 设置作品为开源的,在封二上显示开源地址和作品版本。默认关。
% \end{description}
%
% \subsection{文档信息配置}
% 除了\textsf{\textbackslash maketitle}默认需求的\textsf{\textbackslash title}、
% \textsf{\textbackslash author}与\textsf{\textbackslash date}三条基本信息之外,
% 本文类还提供了以下信息的接口:
% \begin{description}
% \item[\textsf{\textbackslash subtitle}] 副标题信息,建议设置为本作品的英文标题。
% \item[\textsf{\textbackslash typo}] 排版人员信息。
% \item[\textsf{\textbackslash version}] 版本号,在\textsf{opensource}选项开启时有效。
% \item[\textsf{\textbackslash sourcepage}] 开源页面地址,在\textsf{opensource}选项开启时有效。
% \end{description}
% 建议用户在排版时一定要将以上所有的文档信息填全,否则在封二的文档信息盒子里将显示空白。
%
% \subsection{特殊环境}
% 本文类目前提供的特殊环境(命令)有:
% \begin{description}
% \item[\textsf{\textbackslash exercise}] 输出一个表示“练习X”的盒子。
% \item[\textsf{\textbackslash solve}] 引导一个“解”段落。
% \item[\textsf{\textbackslash note}] 引导一个“注记”段落。
% \item[\textsf{\textbackslash analysis}] 引导一个“分析”段落。
% \end{description}
% 请用户善用这些特殊环境,以使得页面趋于标准、清晰。
%
% \subsection{建议}
% 以下建议的提出,大多源于本文档类中未能实现或未做优化的功能。
% 
% \begin{enumerate}
% \item 不建议在本文档类下使用\textsf{part}级别的标题,
% 因本文档类从未定义过\textsf{part}标题对应的样式和目录样式。
% 此外,针对于\QYXF{}目前所排印的书籍而言,这一级的标题无疑
% 是多余了,无论是对作者还是读者而言。
% \item 本文类中,中文采用Adobe系列的字体,英文采用Cambria和Calibri的组合,数学字体采用
% Cambria Math(由\textsf{unicode-math}宏包驱动)。如无特殊情况,不建议用户更换字体。
% \item 请不要自行加载于本文类中已加载过的宏包,避免冲突。
% \end{enumerate}
%
% \StopEventually{\PrintIndex}
%
% \section{qyxf-book 文类实现}
%   以下展示本模板的实现细节。
%
% \subsection{准备工作}
% 首先是必要的声明与对基本文类的加载。
%    \begin{macrocode}
%<*book>
\NeedsTeXFormat{LaTeX2e}[2005/12/01]
\ProvidesClass{qyxf-book}
  [2019/04/28 v1.0 nothing but a skeleton]
\typeout{The book template provided by Qian Yuan Xue Fu}
\LoadClass[10pt,twoside,openany]{book}
%    \end{macrocode}
%
% 接下来定义各关键字参数。
%    \begin{macrocode}
\RequirePackage{kvoptions}
\SetupKeyvalOptions{family=QYXF,prefix=QYXF@,setkeys=\kvsetkeys}
\DeclareBoolOption[false]{parskip}
\DeclareBoolOption[false]{smallersize}
\DeclareVoidOption{b5paper}{\kvsetkeys{QYXF}{smallersize}}
\DeclareBoolOption{color}
\DeclareBoolOption[true]{decoration}
\DeclareBoolOption[false]{opensource}
\ProcessKeyvalOptions*
%    \end{macrocode}
%
% \subsection{设计实现}
% 接下来,由\textsf{ctex}宏包引入对中文的支持,并配置西文字体。
%
%    \begin{macrocode}
\RequirePackage[fontset=adobe,heading]{ctex}
\setmainfont{Cambria}       % 目前默认使用的serif字体
\setsansfont{Calibri}       % 目前默认使用的sans-serif字体
\setmonofont{Courier New}   % 目前默认使用的等宽字体
\RequirePackage{amsmath}
\RequirePackage{unicode-math}
\setmathfont{Cambria Math}  % 与正文字体相适应的数学字体
\renewcommand{\thefootnote}{\textbf{(\arabic{footnote})}}
%    \end{macrocode}
%
% 再引入对于图片、表格的支持、配置。
%    \begin{macrocode}
\RequirePackage{graphicx}
\graphicspath{{./figure/}{./pic/}{./image}}
\setlength{\textfloatsep}{6pt plus 2pt minus 4pt}
\setlength{\intextsep}{6pt plus 2pt minus 2pt}
\RequirePackage{longtable,booktabs}
\RequirePackage{tikz}
\usetikzlibrary{calc,backgrounds}
\RequirePackage{tcolorbox,tabu}
\tcbuselibrary{listings,theorems,skins}
%    \end{macrocode}
%
% 接下来,定义正文所需的一些特殊环境和命令。
%    \begin{macrocode}
\newcommand{\exercise}[1]{\noindent\tcbox[on line,top=0mm,bottom=0mm,%
right=0mm,left=0mm]{\bfseries 练习#1}\ }
\newcommand{\note}{\noindent\textbf{注记}\ }
\newcommand{\solve}{\noindent\textbf{解}\ }
\newcommand{\analysis}{\noindent\textbf{分析}\ }
%    \end{macrocode}
%
% 再来利用\textsf{geometry}、\textsf{hyperref}、\textsf{fancyhdr}等宏包进行页面设计。
% 首先调整页面尺寸。
%    \begin{macrocode}
\RequirePackage{geometry}
\ifQYXF@smallersize
  \setlength{\paperwidth}{176mm}
  \setlength{\paperheight}{250mm}
\else
  \geometry{a4paper}
\fi
\geometry{margin=1in}
%    \end{macrocode}
% 然后是超链接配置。
%    \begin{macrocode}
\RequirePackage{hyperref}
\hypersetup{
    breaklinks,
    unicode,
    linktoc=all,
    bookmarksnumbered=true,
    bookmarksopen=true,
    pdfkeywords={qyxf-book},
    pdfborder={0 0 0},
    linktocpage,
    pageanchor=true
}
%    \end{macrocode}
% 接着,利用\textsf{titlesec}和\textsf{titletoc}宏包进行标题设计,装饰来自于
% \textsf{pgfornament-han}宏包。
%    \begin{macrocode}
\RequirePackage{pgfornament-han}
\ctexset{section={name={\S,}}}
\RequirePackage{titlesec,titletoc}
\titlecontents{chapter}[24pt]{\hspace{-2pc}\filright}
              {\normalsize\bfseries\contentspush{\thecontentslabel\ }}
              {}{\bfseries\titlerule*[8pt]{.}\contentspage}
\titlecontents{section}[20pt]{\filright}
              {\contentspush{\thecontentslabel\ }}
              {}{\titlerule*[8pt]{.}\contentspage}
\titlecontents{subsection}[18pt]{\hspace{2pc}\filright}
              {\contentspush{\thecontentslabel\ }\itshape}
              {}{\titlerule*[8pt]{}\contentspage}
\setcounter{tocdepth}{2}
%    \end{macrocode}
% 最后,设计页眉页脚。
%    \begin{macrocode}
\RequirePackage{fancyhdr}
\fancyhead[EC]{\fangsong\nouppercase\leftmark}
\fancyhead[OC]{\fangsong\nouppercase\rightmark}
\fancyhead[EL,OR]{\bf\thepage}
\fancyhead[ER,OL]{}
\fancyfoot[C]{\textsc{Qian Yuan Xue Fu}\\\resizebox{0.2\linewidth}{1.5ex}{%
\pgfornamenthan[scale=0.05]{58}}}
\fancypagestyle{plain}{%
\fancyhf{}
\fancyfoot[C]{\textsc{Qian Yuan Xue Fu}\\\resizebox{0.2\linewidth}{1.5ex}{%
\pgfornamenthan[scale=0.05]{58}}}
\renewcommand{\headrulewidth}{0pt}
\renewcommand{\footrulewidth}{0pt}
}
\setlength{\headheight}{13pt}
\RequirePackage{lastpage}
%    \end{macrocode}
%
% 其余一些相对次要的功能,暂时未做延伸。
%    \begin{macrocode}
\RequirePackage{enumerate}
\RequirePackage{pifont}
\RequirePackage{appendix}
\ctexset{appendix={number={\Roman{chapter}}}}
\RequirePackage[scale=1.1]{ccicons}
%    \end{macrocode}
%
% \subsection{关键字参数生效处理}
% 以下来处理关键字参数对排版效果的影响。
%    \begin{macrocode}
\ifQYXF@parskip
     \setlength\parskip{0.3\baselineskip}
\fi
%    \end{macrocode}
%
% \subsection{首页制作}
% 下面进行首页和封二的设计。首先,定义若干文档信息:
%    \begin{macrocode}
\newtoks\subtitle
\newtoks\typo
\newtoks\version
\newtoks\sourcepage
%    \end{macrocode}
% 再重定义\textsf{\textbackslash maketitle}命令,以生成首页和封二。
%    \begin{macrocode}
\renewcommand*{\maketitle}{%
\hypersetup{pageanchor=false}
\begin{titlepage}
\phantom{s}
\vspace{3cm}
\begin{center}\large
\begin{tcolorbox}[blanker,borderline west={1mm}{0pt}{black!80},%
left=20pt]
{\fontsize{40pt}{\baselineskip}\selectfont\bfseries\@title}\\[\baselineskip]
{\Huge\itshape\the\subtitle}
\end{tcolorbox}
\begin{tcolorbox}[blanker,borderline west={1mm}{0pt}{black!50},%
left=20pt]
{\Large 作者:\@author\\[0.3\baselineskip]
\@date}
\end{tcolorbox}
\vfill
\texttt{钱学森书院学业辅导中心}\\[0.5\baselineskip]
\textsc{Qian Yuan Xue Fu}\\[0.5\baselineskip]
{XI'AN JIAOTONG UNIVERSITY}
\end{center}
\begin{tikzpicture}[remember picture,overlay]
  \begin{pgfonlayer}{background}
  \node at ($(current page.east) +(0in,0in)$) {%
  \includegraphics[width=0.8\textwidth]{cover.png}};
  \end{pgfonlayer}
\end{tikzpicture}
\end{titlepage}
\thispagestyle{empty}
\newpage
\hypersetup{pageanchor=true}
\phantom{s}\vfill
\begin{tcolorbox}[title={\bfseries 作品信息}]
\ding{228} \textbf{标题:}\@title{} - \textit{\the\subtitle}\\
\ding{228} \textbf{作者:}\@author\\
\ding{228} \textbf{校对排版:}\the\typo\\
\ding{228} \textbf{出品时间:}\@date\\
\ding{228} \textbf{总页数:}\pageref{LastPage}
\end{tcolorbox}
\begin{tcolorbox}[title={\bfseries 许可证说明}]
\centerline{\tcbox{\ccbyncnd\ \fangsong 知识共享 (Creative Commons) BY-NC-ND 4.0 协议}%
\\[0.3\baselineskip]}
本作品采用 \href{https://creativecommons.org/licenses/by-nc-nd/4.0/}{\textbf{CC协议}}
进行许可。使用者可以在给出\emph{作者署名及资料来源}的前提下对本作品进行\emph{转载},但不得对
本作品进行\emph{修改},亦不得基于本作品进行\emph{二次创作},不得将本作品运用于\emph{商业用途}。
\end{tcolorbox}
\newcommand{\opensourceinfo}{根据作者的要求,本作品未公开源码。}
\ifQYXF@opensource
  \begin{tcolorbox}
  本作品已发布于GitHub之上,发布地址为:\\
  \centerline{\the\sourcepage}
  本作品的版本号为\textsf{\the\version}。
  \end{tcolorbox}
\fi
\newpage
}
%    \end{macrocode}
%
% \subsection{目录制作}
% 下面稍稍重定义目录的生成方式。
%    \begin{macrocode}
\renewcommand\tableofcontents{%
\pagestyle{empty}
\centerline{%
\normalfont\LARGE\bfseries\contentsname%
\@mkboth{\MakeUppercase\contentsname}{\MakeUppercase\contentsname}
}
\vskip 3ex%
{\setlength\parskip{0pt}\@starttoc{toc}}
\cleardoublepage
\pagestyle{fancy}
\setcounter{page}{1}
}
%</book>
%    \end{macrocode}
% 到此,文档类的定义正式结束。
%
% \section{qyxf-note 文类实现}
%    \begin{macrocode}
%<*note>
\NeedsTeXFormat{LaTeX2e}
\ProvidesClass{qyxf-notice}[2019/05/10 Template for qyxf's note]
\LoadClass[12pt,oneside]{article}
%</note>
%    \end{macrocode}
%
% \section{qyxf-notice 文类实现}
%    \begin{macrocode}
%<*notice>
\NeedsTeXFormat{LaTeX2e}
\ProvidesClass{qyxf-notice}[2019/04/17 Template for qyxf's notice]
\LoadClass[12pt,a5paper,onecolumn,oneside]{article}

\RequirePackage[margin=1cm,bottom=2.5cm]{geometry}
\RequirePackage[fontset=adobe]{ctex}
\RequirePackage{fontspec}
\setmainfont{Cambria}

\RequirePackage{graphicx}
\graphicspath{{images/}}
\RequirePackage{pgfornament}

\RequirePackage{fancyhdr}
\fancyhf{}
\renewcommand{\headrulewidth}{0pt}
\cfoot{%
\hrule\vskip\baselineskip
\centering\includegraphics[width=0.55\textwidth]{name.png}
}

\renewcommand{\maketitle}{
\centerline{\Large\fbox{\itshape QyxF$\ \spadesuit$} {\kaishu 钱学森书院学业辅导中心}}%
%\vskip0.2\baselineskip
\begin{center}
\begin{tikzpicture}
\node (A) at (0.2\textwidth,0) {};
\node (B) at (0.8\textwidth,0) {};
\draw (A) to [ornament=88] (B);
\end{tikzpicture}
\end{center}
\vskip0.5\baselineskip
\centerline{\LARGE\bfseries\@title}
\vskip\baselineskip
\thispagestyle{fancy}
}
%</notice>
%    \end{macrocode}
% \Finale